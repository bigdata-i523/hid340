\RequirePackage[hyphens]{url}
\documentclass[sigconf]{acmart}
\raggedbottom
\usepackage[english]{babel}
\usepackage{blindtext}

\input{format/final}

\begin{document}

\title{BigchainDB: A Big Database for the Blockchain?}
\author{Timothy A. Thompson}
%\orcid{0000-0001-6574-9010}
\affiliation{%
  \institution{Indiana University Bloomington}
  \streetaddress{School of Informatics, Computing, and Engineering}
  \city{Bloomington} 
  \state{Indiana} 
  \postcode{47408}
}
\email{timathom@indiana.edu}

\begin{abstract} 
Decentralized systems such as Bitcoin, the Interplanetary File System, and Ethereum have been designed with the intention of reengineering the architecture of online information networks, of minimizing exposure caused by centralized points of failure, and of creating new social models for the exchange of data---which is posited as a valuable asset in and of itself. Are these kinds of systems also able to support big data analytics and processing? If so, what stands to be gained by taking a blockchain-based approach to big data? Efforts to integrate blockchains into big data pipelines must inevitably address the tradeoff between security and scalability. BigchainDB is a new decentralized database framework that adds blockchain-based features, such as immutability and secure asset management, to traditional NoSQL distributed databases. Although it is still in the early stages of development, BigchainDB promises to make a significant contribution to the ways in which data is shared, processed, and managed at scale.
\end{abstract}

\keywords{i523, HID340, Decentralization, Databases, NoSQL, Blockchains, BigchainDB}

\maketitle

\section{Introduction}
Approaches to managing and processing big and complex data, such as the Lambda Architecture framework, have stressed the importance of treating data as an immutable asset \cite{nM15}. In this view, data should never be updated, but only appended. In systems that allow data to be deleted or updated, big data only amplifies the surface of exposure to human error, and systems that conform to the standard relational database model of incremental updates become increasingly brittle as the scale of data increases \cite{nM15}. Inadvertent deletions can trigger a cascade of data loss and system disruption that can be particularly difficult and costly to recover from. Even those who have criticized the specifics of the Lambda Architecture model (which proposes a complex internal division, within big data systems, between a batch layer and a realtime layer) agree that data immutability is an important foundation for building massively scalable platforms \cite{jK14}.

In decentralized, blockchain-based systems such as Bitcoin, immutability takes on an even more critical role. Without immutability and concomitant mechanisms such as Merkle tree hashing, it would not be possible to verify Bitcoin transactions for authenticity, nor would it be possible to maintain the ``trustless'' nature of the network---which is what allows decentralization itself to succeed \cite{aA17}. In addition to Bitcoin, the emerging decentralized data ecosystem currently comprises platforms for computation (Ethereum) and file storage (Interplanetary File System---IPFS), but database software for managing metadata about assets is still lacking. BigchainDB is a databse solution that has been designed to fill this niche \cite{bigDB16, tMBI15}.

\section{Distributed versus Decentralized}
Once it has been elevated to a core piece of system architecture, data immutability can become either a burden or an opportunity. The Lambda Architecture model leverages immutability within the context of an internally distributed environment, using storage solutions such as the Hadoop Distributed File System (HDFS) for processing on the batch layer \cite{nM15}. Distributed systems are not the same as decentralized systems, however, and the terminology itself can be misleading, as illustrated by Baran's often-reproduced work depicting the continuum between centralized and distributed networks (shown in Figure \ref{f:baran}). In Baran's model, decentralized networks are vulnerable to attack because of their reliance on hubs, whereas distributed networks are more durable because they employ a resilient grid-like structure \cite{pB64}. In the context of the current discussion, ``decentralized'' systems such as Bitcoin are in fact exemplars of distributed models of connectivity. Distributed file systems such as HDFS, by contrast, may exist within highly centralized platforms or services. Here, the term decentralized will be used to refer to systems that embrace distributed models of organization both internally and externally. 

\begin{figure}
\includegraphics[width=1.0\columnwidth]{images/baran-fig1.png}
\caption{Baran's centralized--distributed network continuum \cite{pB64}}
\label{f:baran}
\end{figure}

\subsection{Tradeoffs between Security and Scalability}
Bitcoin's high level of security and resistance to attack are appealing features to engineers concerned with issues of data integrity. However, the Bitcoin network and blockchain are currently not equipped to manage big data \cite{kC16}. Compared to commercial financial transaction processors, which are capable of processing thousands of transactions per second, Bitcoin's computationally intensive ``Proof of Work'' model limits the network's throughput to a maximum of about 7 transactions per second \cite{kC16}. Recommendations for improving the scalability of distributed ledgers such as Bitcoin range from adjusting system parameters (for example, maximum block size) to sharding the transaction validation layer in order to take advantage of parallel processing; however, the need for community approval and adoption of scalability solutions means that changes will take time to be implemented \cite{kC16}.

\subsection{Blockchains for Big Data}
The Bitcoin blockchain could itself be viewed as a globally distributed database, albeit not a particularly efficient or effective one. What is the primary benefit, then, of attempting to bring blockchain-based technologies to bear on big data? The potential value of integrating blockchains into big data pipelines is extrinsic rather than intrinsic: blockchains do not enable new analytical frameworks or algorithms per se, but rather promise to revolutionize the economies of exchange that determine the value and availability of big data \cite{tMBD16}. Nevertheless, in order for big data pipelines to be ``blockchainified,'' the issue of scalability must be addressed. Because a systematic reengineering of blockchain systems such as Bitcoin seems unlikely in the near term, an alternative approach would be to modify existing big data systems by wrapping them with a blockchain layer. The latter approach is the one that has been adopted by the designers of BigchainDB \cite{bigDB16}.

\section{BigchainDB}

\blindtext
\blindtext

\blindtext

\blindtext
\blindtext

\blindtext
\blindtext

\section{Conclusion}

\blindtext
\blindtext

\begin{acks}
The author would like to thank Dr. Gregor von Laszewski and the i523 teaching assistants for their support and suggestions in writing this paper.
\end{acks}

\bibliographystyle{ACM-Reference-Format}
\bibliography{report} 

\end{document}

























